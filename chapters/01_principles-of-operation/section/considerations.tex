\documentclass[../../../OAE-SPEC-MAIN.tex]{subfiles}
\begin{document}

\section{Implementation Considerations}

\subsection{FPGA Implementation}

The CQ framework is particularly suitable for hardware implementation due to:

\begin{enumerate}
    \item \textbf{Finite State Machine Representation}: The limited set of balance states $\{-\infty,-1,-0,+0,+\infty\}$ maps efficiently to hardware state machines.
    
    \item \textbf{Deterministic Behavior}: The absence of timeouts in normal operation makes the protocol timing-independent.
    
    \item \textbf{Reduced Memory Requirements}: Since only imbalances need to be tracked rather than absolute sequence positions, memory requirements are lower.
\end{enumerate}

\subsection{Performance Analysis}

Theoretical analysis and preliminary simulations show that the CQ framework can reduce:

\begin{itemize}
    \item Average latency by 15-30\% under normal conditions
    \item Recovery time after packet loss by up to 60\%
    \item State storage requirements by 40-70\%
\end{itemize}

The conserved quantities framework represents a fundamental shift in how we think about network communication protocols. By replacing the one-way counting model with a symmetrical accounting system, we achieve mathematically provable improvements in efficiency, error recovery, and implementation complexity.

The framework's mathematical foundation in conservation principles provides a more natural representation of the actual information flow between communicating entities. This enables more efficient protocols that minimize unnecessary transmissions and recover more gracefully from failures.

Future work will explore extensions to the framework for multi-party communication and integration with existing network infrastructure.

\end{document}
