\documentclass[../../../OAE-SPEC-MAIN.tex]{subfiles}
\begin{document}



\section{Fault Model}

\marginnote{
Benefits include (i) Shorter packets and more effective use of bandwidth, (ii) more complete coverage of possible failure modes. (iii) Guarantees at least the first slice is perfect (matches what the transmitter knows they sent). }

AE-Links present two major differences to the conventional FEC thinking in today's Ethernet, which exploits the physics from 25Gb/s to 1.6Tb and beyond:

\begin{description}
\item [Perfect Information Transfer (PIF)] Æ-Links use Back-to-Back (B2B) Shannon Links, where the receiver returns the first 8-byte slice of each 64-Byte packet to the transmitter. This ``here is what I heard you say" ( Perfect Information Transfer (PIF)\cite{Paper from Garner}

\item [Epistricted Registers (EPI)] Borrowing from the Spekkens' toy model for quantum entanglement, we narrow down the possible entangled states to a vastly smaller set of possibilities, using the model described in Quantum Ethernet\cite{Quantum Ethernet}.
\end{description}



\subsection{Failure Model}

Consider a network of \(n\) nodes connected by undirected Ethernet
links.  Each link can be in one of four independent reliability
states,
\marginnote{
\[
\Sigma=\{00,01,10,11\},
\]}
where \(11\) means the link works in both directions, \(10\) or
\(01\) means it works in only one direction, and \(00\) means it
is broken in both directions.

%\subsection{Link count}

Because every node may attach to at most eight neighbours (an
\emph{octavalent} mesh), the number of physical links is
\marginnote{
\[
L(n)=\min\!\bigl\{\tbinom{n}{2},\,4n\bigr\}
=
\begin{cases}
\binom{n}{2}, & n\le 9,\\[6pt]
4n,            & n\ge 9.
\end{cases}
\]}

%\section{Reliability configurations}

Each link chooses a state from \(\Sigma\) independently, so the total
number of configurations is \(4^{\,L(n)}\).
Exactly one of these is fully healthy (all links in state \(11\)), hence
\[
\text{FailureModes}(n)=4^{\,L(n)}-1.
\]

\subsection{Enumerated results for \(2\le n\le20\)}

%\begin{table}[ht]
\begin{margintable}
\centering
\begin{tabular}{@{}rrr@{}}
\toprule
\(n\) & \(L(n)\) & Failure modes \(4^{\,L}-1\)\\
\midrule
 2 &  1 & 3\\
 3 &  3 & 63\\
 4 &  6 & 4\,095\\
 5 & 10 & 1\,048\,575\\
 6 & 15 & \(1.074\times10^{9}\)\\
 7 & 21 & \(4.398\times10^{12}\)\\
 8 & 28 & \(7.206\times10^{16}\)\\
% 9 & 36 & \(4.722\times10^{21}\)\\
%10 & 40 & \(1.209\times10^{24}\)\\
%11 & 44 & \(3.095\times10^{26}\)\\
%12 & 48 & \(7.923\times10^{28}\)\\
%13 & 52 & \(2.028\times10^{31}\)\\
%14 & 56 & \(5.192\times10^{33}\)\\
%15 & 60 & \(1.329\times10^{36}\)\\
%16 & 64 & \(3.403\times10^{38}\)\\
%17 & 68 & \(8.711\times10^{40}\)\\
%18 & 72 & \(2.230\times10^{43}\)\\
%19 & 76 & \(5.709\times10^{45}\)\\
%20 & 80 & \(1.462\times10^{48}\)\\
\bottomrule
\vspace{8pt}
\end{tabular}
\caption{Failure‑mode counts for an octavalent mesh with \(n\) nodes.}
\vspace{12pt}
\end{margintable}
%\end{table}


\end{document}
