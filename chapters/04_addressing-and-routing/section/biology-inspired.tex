\documentclass[../../../OAE-SPEC-MAIN.tex]{subfiles}

\begin{document}

\section{Ants, Bees, Snakes, Spiders, and Worms: Biologically Inspired Topology Learning}

To support adaptive and scalable routing in chiplet interconnects, we propose several biologically inspired methods for topology discovery and maintenance, especially for networks with nodes of valency 4, 6, or 8.

\subsection{Ant-Based Discovery}

\begin{itemize}
\item \textbf{Scouts:} Explore the network, collecting path metrics.
\item \textbf{Pheromone Trails:} Nodes cache recent path qualities.
\item \textbf{Reinforcement:} Successful paths are promoted; old paths decay.
\item \textbf{Usage:} Enables adaptive routing without global knowledge.
\end{itemize}

\subsection{Bee-Inspired Exploration}

\begin{itemize}
\item \textbf{Hive Nodes:} Aggregate partial topological information.
\item \textbf{Scouts:} Sample paths and report back.
\item \textbf{Dance Protocols:} Broadcast high-quality routes to other nodes.
\item \textbf{Usage:} Useful for semi-centralized routing optimization.
\end{itemize}

\subsection{Snake Traversals}

\begin{itemize}
\item \textbf{Traversal Packet:} Systematically visits each reachable node.
\item \textbf{Data Aggregation:} Builds complete network maps.
\item \textbf{Return Path:} Reports findings back to the origin node.
\item \textbf{Usage:} Suitable for diagnostics or rare full-network verification.
\end{itemize}

\subsection{Spider Web Construction}

\begin{itemize}
\item \textbf{Threads:} Discovery packets sent on all ports.
\item \textbf{Local Web:} Maintains neighbor lists and multi-hop connections.
\item \textbf{Tension Metric:} Indicates link quality (latency, congestion).
\item \textbf{Usage:} Builds resilient, redundant local meshes.
\end{itemize}

\subsection{Wormhole Routing: Worm Behavior as Forwarding Strategy}

Wormhole routing segments packets into flits and forwards them in a pipeline manner.

\subsection{Comparison with Other Techniques}

\begin{itemize}
\item \textbf{Store-and-Forward:} Full packet buffered at each hop; simple but high-latency.
\item \textbf{Cut-Through:} Begin forwarding upon header arrival; requires moderate buffer sizes.
\item \textbf{Wormhole:} Forward flits immediately, with minimal buffers.
\end{itemize}

\subsection{Mechanics of Wormhole Routing}

\begin{itemize}
\item \textbf{Head Flit:} Reserves path.
\item \textbf{Body Flits:} Stream through reserved path.
\item \textbf{Tail Flit:} Releases resources.
\end{itemize}

\subsection{Hardware Considerations}

\begin{itemize}
\item \textbf{Buffers:} Minimal per-port buffering (1--2 flits).
\item \textbf{Flow Control:} Credit-based or handshake-based.
\item \textbf{Stall Propagation:} Congestion can block entire worm.
\item \textbf{Virtual Channels:} Prevent deadlocks and increase concurrency.
\end{itemize}

\subsection{Integration Strategies}

\begin{itemize}
\item \textbf{Ant + Wormhole:} Explore paths via ants, forward data via wormhole flits.
\item \textbf{Snake + Destination-Based:} Use full traversals to populate local forwarding tables.
\item \textbf{Spider + Name-Based:} Webs retain service-to-node mappings.
\end{itemize}

\end{document}