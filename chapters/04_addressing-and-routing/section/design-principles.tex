\documentclass[../../../OAE-SPEC-MAIN.tex]{subfiles}

\begin{document}


\section{Design Principles}
\marginnote{\footnotesize\textit{Came from KAO-DAE-Specifications/Sections/DesignPrinciples.tex}}

The Daedaelus Substructure (DS) is made of Cells (physical nodes, complete with processor, memory, and FPGA-Enabled SmartNICs), and physical Links which run the DSPS Protocol, and are based on the following principles:

\begin{description}

	\item [Every cell has a constrained $7 \pm 2$ ports per cell]\footnote{Typical servers today have from 6-8 ports; with an additional 2 ports are used for connecting to Top of Rack (ToR) switches}, %and the 3rd or 5th port is used as an ``out of band'' (OOB) network for provisioning, maintenance, and security.} 
	which balances the need for enough path diversity between any pair of cells to make partitions extremely rare while bounding the complexity of the algorithms for routing, healing and recovery.  However, the most significant advantage of a constrained number of ports (valency) is in the building of long-lived Treez and their extended paths (\href{https://en.wikipedia.org/wiki/Dendrite}{dendrites}) which can be used for self-organization\footnote{For example, for migrating data and computations automatically, based on local only information, to create chains or gradients of, for example, data temperature, or the automatic layout of graph computations, or evolving topologies for deep learning applications.}. 

	\item [Local information only] Each cell has information only about itself and the cells its ports are connected to.  While this limitation rules out certain global optimizations, it enhances flexibility in that only neighbors of a cell need deal with changes made to that cell.  For example, adding a link only affects the two cells it connects. Local Observer View (LOV).

	\item [Failure detection with DSPS]  provides Temporal Intimacy (The \texttt{TIK\hspace{1pt}TYK\hspace{1pt}TIK}, The I Know That You Know That I Know property).   \texttt{TIK\hspace{1pt}TYK\hspace{1pt}TIK} simplifies the enforcement of desired properties, such as in-order message delivery (even when a failure forces subsequent messages to take a different route to their destination), and Atomic Information Transfer (AIT).

	\item [Event driven] The network fabric is \emph{event driven}, \emph{i.e.}, there are no network-specific timeouts. Self-sustaining events are created in the \texttt{links} using the ENTL mechanism. This is in contrast to conventional computing, where all events are driven by processes running on one or more cores of a processor.

	\item [Every cell builds a spanning tree with itself as root] A spanning tree provides certain guarantees.  Each root can reach all other cells in the fabric with a \emph{tree cast}, and each cell can reach the root with a \emph{root cast}.  In addition, it is easy to enforce in-order delivery of messages between any pair of cells, simplifying many distributed computing algorithms.

	\item [Each link keeps state needed to recover from routing around its failure] Link state is maintained by the two ports it connects to.  This information is used to reestablish the ordering guarantees when routing around the link if it fails, to maintain complimentary AIT state on each end of the link, and to silently maintain ``presence'' for the two parties.

	\item [Each cell keeps state to aid in routing around failures] The spanning trees (black trees) are built on the \texttt{groundfabric} (the physical connectivity), and all cell trees together represent the \texttt{groundplane} which is a connected undirected logical graph that matches the underlying \texttt{groundfabric} exactly.  When building the spanning trees, each cell records the state of each port for each tree in the fabric.  For each tree one port points to a parent, others point to zero or more children, and the rest are not part of the tree.   These ports have been \emph{pruned}.  We call this state a \emph{TRAPH}, for Tree-Graph.  When the link connected to the port pointing to a parent fails, the cell uses the TRAPH data to find an alternate port to reach the new parent.

	\item  [\texttt{groundplane} is the lowest sub-level of the FPGA Substructure]: the physical layer.
All TRAPHs (selected graph covers) are built as subsets of the \texttt{groundplane}. Each TRAPH has a controlling tree (the owner), based on the cell which created and named the TRAPH. This owner cell may keep itself as the sole entity which controls the TRAPH,  but more typically, it will select one or more other cells to provide failover should it fail. The failover protocols (Tree Paxos) are described in a separate section of this document.
	
\end{description}

A recursively stacked set of logical TRAPHs may be built on the \texttt{groundplane} to provide the logical segregation layers for secure provisioning and confinement for management entities, such jurisdictions, tenants and sub-tenants. These will often be referred to as the \emph{grey} layers, because they are in the substructure, and are hidden from the  hypercontainers above (which run on the other side of the PCIe bus in the hose processor).

Virtual (Colored) TRAPHs may be stacked on top of any of the logical gray layer TRAPHs.  Virtual TRAPHs are available under API control for the operating systems and applications above.



\subsection{Relative Addressing}

A major distinction from conventional data centers is that ECNF addressing is based on cell-to-cell (C2C) direct connections. Every communication from an agent on a cell is to one or more ports. The message is sent to the port on the other side of the link, which can forward the message on one of that cell's ports just as switches do. 

This addressing scheme has a number of advantages.  For example, each cell needs to maintain limited routing information.  Should the path to some target agent change, perhaps because it migrated to another cell or due to a failover, only a limited number of cells need to update their routing information.  Additionally, an attacker who penetrates a cell can only address its direct neighbors, limiting any potential damage

\subsection{Port Labelling} 

Some failover algorithms use path information in the form of a sequence of port identifiers between some cell and the root.  Each cell is free to assign arbitrary labels to its ports as long as each of its ports is assigned a label unique to the cell.  However, debugging and network management will be easier if a consistent convention is used, as shown in Figure~\ref{fig:foo} for a cell with 12 ports.
 
\fig[width=0.5\textwidth]{HexagonPortIdentifiers2.pdf}[Port Labeling convention: 6--12 ports][fig:HexagonPortIdentifiers2]
 
 
\subsection{Hop count}

Knowing the number of hops from a given cell to the root can help in selecting shorter failover paths.  However, there is a cost.  When the path from a cell to the root changes, other cells in the TRAPH may need to be notified to update their hop counts.

\subsection{Path information}

When a link breaks, LW must find a new path to the root of every tree that uses the now broken link.  Hop count helps guide the selection of which port to try next, but the new path must be tested all the way to the root.  Keeping path information in the form of the each cell's port identifier on the path from the root can reduce that cost.  As with hop count, other cells in the TRAPH must be notified to update their path information following a failover.

While messages carrying this information grow with the length of the path, the size is manageable.  If we use four bits to label each port, and we use 1K of a 1,500-byte packet for path data, then we can manage 2,000 element paths with a single packet.  With balanced trees in a datacenter with $N$ cells, the longest path will be $\log_8{N} = \frac{1}{3}\log_2{N}$, which is far less than 2,000 for any datacenter envisioned today.  Even with unbalanced trees grown stochastically, the longest path is likely to be $O(\sqrt{N})$.  We can choose to truncate the path information at the cost of slightly somewhat more attempts before finding a new path.

Introducing Kleinberg links makes the problem tractable even with 64-byte packets, since the average hop count is around 5.  Of course, some paths will be longer, but taking 10 bytes of the packet for path information allows us to handle up to 20 hops.  We can truncate the path information of the few paths that are longer.

\subsection{Packet-level Time Reversal}

When a recipient of a packet associated with an AIT token cannot pass the packet to the next higher level in the stack, it implements local \emph{time reversal}.  The result is that the cells on the two sides of the link end up in the same state they were in before the packet in question was sent.  This procedure can be extended to any number of intervening nodes.

This section is premised on the fact that we want to do this kind of time reversal when a link breaks, but I'm not sure why.  It would seem that LW only needs to know if it should send the last packet sent before the failure to J, the join node responsible for reestablishing message ordering.  No time reversal needed.


\end{document}

