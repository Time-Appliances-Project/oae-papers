\documentclass[../../../OAE-SPEC-MAIN.tex]{subfiles}

\begin{document}




\section{Introduction}
\marginnote{\footnotesize\textit{Came from ./AE-Specification-ETH/standalone/Graph-Algorithms.tex}}

This document proposes a Layer 2 Ethernet routing protocol called \emph{Scouting at Layer 2}, inspired by recent advances in edge coloring algorithms from graph theory. It aims to enable deterministic, loop-free path discovery and forwarding solely at the MAC layer, bypassing the need for Layer 3 mechanisms such as IP routing.

Ethernet is traditionally a broadcast-based Layer 2 protocol, relying on IP-based Layer 3 protocols for routing. However, modern data centers and specialized networks demand low-latency, deterministic, and topology-aware communication mechanisms without the full overhead of the IP stack. To this end, we introduce \emph{Scouting at Layer 2}, a distributed routing protocol that operates entirely within the Data Link Layer.

The design is motivated by recent work on edge coloring of graphs in near-linear time \cite{li2024vizing}, which provides a scalable and collision-free scheme for link differentiation.

\section{Design Goals}

\begin{itemize}[label=--]
    \item \textbf{Layer 2 Only}: Operates exclusively using MAC addresses.
    \item \textbf{No Broadcast Storms}: Avoids STP/RSTP flooding and enables deterministic paths.
    \item \textbf{Loop-Free Forwarding}: Guarantees no cycles using path identifiers.
    \item \textbf{Dynamic Topology Support}: Accommodates changes in network structure.
    \item \textbf{Low Computational Overhead}: Efficient enough to run in SmartNICs or ASICs.
\end{itemize}

\section{Graph-Theoretic Inspiration}

In the protocol, each Ethernet node and its direct connections are modeled as a graph $G = (V, E)$, where:

\begin{itemize}
    \item Vertices $V$ are MAC-layer devices (bridges, switches, NICs).
    \item Edges $E$ represent Ethernet links.
    \item Each edge is assigned a \textbf{color}, i.e., a unique local forwarding tag, such that no two edges incident to the same vertex share the same color.
\end{itemize}

Using the result of Li et al.~\cite{li2024vizing}, we can perform this coloring with at most $\Delta+1$ colors in $O(m \log \Delta)$ time, where $m = |E|$ and $\Delta$ is the maximum degree.

\end{document}

