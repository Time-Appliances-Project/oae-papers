\documentclass[../../../OAE-SPEC-MAIN.tex]{subfiles}

\begin{document}


\section{Protocol Description}

\subsection{Initialization Phase}

Each node performs neighbor discovery and assigns temporary colors (tags) to its outgoing links, ensuring local uniqueness. Nodes then gossip their tag assignments to neighbors until a global stable coloring is reached.

\subsection{Path Discovery Phase}

To reach a given MAC address, a node constructs a sequence of tags (path identifiers) describing a color-consistent path through the network. These path sequences are constructed in a Dijkstra-like traversal with color-awareness to avoid collisions.

\subsection{Frame Forwarding}

Frames are modified to include a \emph{Path Identifier Sequence (PIS)} field, which encodes the list of edge colors (tags) to follow. As the frame traverses the network:

\begin{enumerate}
    \item The switch reads the next tag in the PIS.
    \item It matches this tag to an outbound port.
    \item It decrements the PIS and forwards the frame.
\end{enumerate}

This process continues until the PIS is empty, and the destination MAC is reached.

\end{document}

