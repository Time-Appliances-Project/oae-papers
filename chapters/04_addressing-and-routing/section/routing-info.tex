\documentclass[../../../OAE-SPEC-MAIN.tex]{subfiles}

\begin{document}


\section{Frame Format}

\vspace{6pt}
\begin{tabular}{@{}lll@{}}
\toprule
\textbf{Field}             & \textbf{Length (Bytes)} & \textbf{Notes} \\
\midrule
Destination MAC           & 6                      & Standard MAC \\
Source MAC                & 6                      & Standard MAC \\
Type/PIS Identifier       & 2                      & Ethertype or custom \\
Path Identifier Sequence  & Variable               & Encoded tag list \\
Payload                   & Variable               & As usual \\
FCS                       & 4                      & Standard CRC \\
\bottomrule
\end{tabular}
\vspace{12pt}

\section{Advantages}

\begin{itemize}
    \item No need for Layer 3 routing tables.
    \item Supports programmable switching (e.g., in SmartNICs or eBPF).
    \item Scales to large networks with sparse connectivity.
    \item Deterministic pathing avoids congestion and loops.
\end{itemize}

\section{Challenges}

\begin{itemize}
    \item Path sequence length is limited by MTU.
    \item Requires coordination to avoid inconsistent tag assignment.
    \item Topology changes require propagation of new path info.
\end{itemize}

\section{Applications}

\begin{itemize}
    \item HPC clusters with low-latency mesh topologies.
    \item Edge compute zones with fixed link-layer infrastructure.
    \item Datacenter overlays where IP is inefficient or unavailable.
\end{itemize}

\section{Conclusion}

By leveraging edge-coloring strategies for deterministic path discovery and forwarding, Scouting at Layer 2 offers a novel approach to MAC-layer routing. Its foundation in recent algorithmic advances such as those by Li et al.~\cite{li2024vizing} provides a robust and efficient scheme, particularly suited to programmable and high-performance environments.

\end{document}

