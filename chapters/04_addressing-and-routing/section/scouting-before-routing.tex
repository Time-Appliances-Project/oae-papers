\documentclass[../../../OAE-SPEC-MAIN.tex]{subfiles}

\begin{document}

\section{Biologically Inspired ``Scouting'' before ``Routing''}
\marginnote{\footnotesize\textit{Came from ./AE-Specifications-Eth/Biology.tex}}
%
\subsection{Local decisions and emergent global organization}

\begin{itemize}
\item Scouting/Discovery Phase: Biologically inspired methods (e.g., ant-colony-inspired or pheromone-based algorithms) often employ “scout” packets or “explorer” agents that roam the network. These scouts collect local congestion or path-quality information and deposit some form of “trail” (akin to pheromones).
\item Emergent Routing Table Updates: Each router or switch updates local routing information (sometimes called a local “pheromone table”). Over time, paths that prove consistently ``good'' get reinforced; less efficient paths fade. This local, probabilistic approach can converge on globally efficient routes with no central coordination.
\end{itemize}

\subsection{Relevance to On-Chip or 2D Mesh Topologies}

\begin{itemize}
\item Local Compass Directions: In a regular mesh (e.g., 2D grid) or torus, each router has up to 4 (N, E, S, W) or 8 ports (adding NW, NE, SW, SE). A biologically inspired algorithm can treat each output port as a possible ``direction of travel.''
\item Natural Fit for Scouting: The local directional structure matches how “ants” or “foraging agents” might look around in each direction, choosing a route based on local pheromone levels (akin to local congestion or link utilization).
\end{itemize}

Thus, the scouting/discovery mechanism is all about gathering local ``pathworthiness'' data and then directing future traffic toward better routes—exactly how a local compass-based system can easily be integrated.

\subsection{Bufferless (Hot-Potato) Routing}

\subsection{Basics of Bufferless Routing}

\begin{itemize}
\item No Packet Buffers (or Very Limited Buffers): In a bufferless architecture, every router typically either immediately forwards or deflects each incoming packet. Packets cannot wait in large queues when an output port is congested.
\item Hot-Potato / Deflection Character: When the preferred output port is unavailable, the packet is sent out of a different (less ideal) port—``hot-potato'' style—rather than being buffered.
\end{itemize}

\subsection{Connection with Biologically Inspired Approaches}
\begin{itemize}
\item Continuous Movement: Biologically inspired scouts are already designed to wander and discover; in a bufferless system, “wandering” (via deflections) is also central. This synergy means a router can apply a heuristic (like a pheromone table) to pick the “best available port” quickly, but if that port is busy, the packet must choose an alternate direction.
\item Adaptive Reinforcement Over Time: In a bufferless design, a packet cannot linger while waiting for the optimal output. However, local “pheromone” or “congestion” metrics can still help route the majority of packets down better ports more often. Over time, high-traffic edges might become less appealing, guiding packets to less-congested directions.
\end{itemize}

\subsection{Deflection Routing}

\subsection{How Deflection Routing Works}

\begin{itemize}
\item Forced Misrouting / Deflection: If the desired or minimal-distance output port cannot be taken (due to contention), the router picks another output. The packet may travel away from its ultimate destination (a “deflection”), but eventually, it should be re-routed back on track.
\item Common in Low- or No-Buffer Architectures: Deflection routing is one way to handle resource contention when buffer space is unavailable.
\end{itemize}
%
\subsection{Tying It Back to the Compass Ports (N, E, S, W, NW, NE, SW, SE)}
%
\begin{itemize}
\item Local Prioritization: In an 8-port (or 4-port) router, one can define a strict or heuristic priority among the directions. For example, a packet traveling generally “north-east” might prefer the N or E port if free; if both are busy, it might deflect NE, or in the worst case, deflect NW or SE.
\item Biologically Inspired Ranking: The “pheromone” concept can be used to rank the output directions. The highest “pheromone” port is tried first, then so on down the rank. This effectively merges a local heuristic (pheromone) with forced deflection for whichever ports remain free.
\end{itemize}

In practice, such a scheme allows packets to “scout” and reinforce certain directions while still ensuring that they never have to wait for a blocked port.

\subsection{Example Flow in an 8-Port Router}

\begin{enumerate}
\item		Receive a Packet coming in from, say, the south port.
\item		Look Up Destination (or partial coordinate heading). For instance, the packet is trying to reach a node in the north-east region, so N or E might be favored.
\item		Check Local “Pheromone” or Routing Table: Suppose the local pheromone table says port NE is the best guess based on past traffic patterns.
\item		If NE Port Is Free: Forward the packet NE.
\item		If NE Port Is Busy: Check next best local direction (N, E, or NW/SE fallback).
\item		If All Preferred Ports Are Busy: Packet is deflected to any open port (could be even SW in the worst case).
\item		Local Table Update: The router sees how that choice ended up affecting the packet (if it eventually left the region quickly or ended up in a congested area). Over time, these experiences feed back into local pheromone levels.
\end{enumerate}

Despite the forced misrouting (deflections), the biologically inspired feedback approach often keeps net throughput healthy and tries to avoid systematic congestion ``hot spots.''

\subsection{Connection with the Literature}%and Further Reading

\begin{enumerate}
\item 	1.	Hot-Potato Routing (Deflection Routing):
	\begin{itemize}
	\item Baran, P. (1962). On Distributed Communications Networks. IEEE Transactions on Communications. (Early ideas of “hot-potato” and distributed routing).
	\item Dally, W., \& Towles, B. (2004). Principles and Practices of Interconnection Networks. (Excellent overview of deflection routing in modern network design).
 	\end{itemize}

\item 	2.	Biologically Inspired / Ant-Based Routing:
	\begin{itemize}
	\item Di Caro, G. A., \& Dorigo, M. (1997). AntNet: Distributed stigmergetic control for communications networks. Journal of Artificial Intelligence Research.
	\item Schoonderwoerd, R., Holland, O., Bruten, J., \& Rothkrantz, L. (1996). Ant-based load balancing in telecommunications networks. Adaptive Behavior.
	\end{itemize}
\item	3.	Network-on-Chip with Deflection/Bufferless Approaches:
	\begin{itemize}
	\item Moraes, F. et al. (2004). A Low Area Overhead Packet-switched Network on Chip: Architecture and Prototyping. SBCCI.
	\item Fallin, C., et al. (2012). CHIPPER: A Low-Complexity Bufferless Deflection Router. HPCA.
	\end{itemize}
\end{enumerate}

These resources flesh out how bufferless or deflection routing is implemented (especially in on-chip contexts) and how biologically inspired heuristics can be adapted to local, minimal-knowledge scouting decisions.

\subsection{Concluding Remarks}

\begin{itemize}
\item Shared Tenets: Both biologically inspired scouting and deflection-based, bufferless routing rest on local decision making. In biologically inspired schemes, scouting packets “discover” or “reinforce” certain paths. In deflection routing, each router makes a quick (local) decision when a preferred port is blocked, forcing packets to keep moving.
\item Complementary Mechanics: Because biologically inspired “pheromone” updates naturally reflect congestion and path usage, they integrate well with a bufferless or deflection style—turning forced misroutes into valuable “exploration” signals that feed back into local heuristics.
\item Directional Routing: The presence of N, S, E, W (plus diagonals) simply defines how many possible local moves each node (router) can attempt. In 2D meshes or tori, these directions make for a convenient coordinate system that parallels how ants (or other scouts) might sense local gradients or pheromone intensities in each of eight compass directions.
\end{itemize}

Overall, if we combine a scouting mechanism (to adaptively find neighbors and good routes) with deflection routing (to handle buffer constraints or high contention), we get a dynamic, emergent routing system in which packets flow continuously and local updates shape global traffic patterns in a self-organizing fashion.

All this happens without the need for Source/Destination Addresses, which present severe security problems by exposing the ``identity" of nodes making them vulnerable to attack.


\end{document}
