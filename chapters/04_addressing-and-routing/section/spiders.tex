\documentclass[../../../OAE-SPEC-MAIN.tex]{subfiles}

\begin{document}

\section{4. Spiders: Building and Maintaining “Webs” of Connectivity}

Spiders create webs that dynamically adapt to external stresses or breaks. A “web-based” approach for chiplet networks focuses on building a resilient mesh:

\begin{enumerate}
	\item  	Web Construction: Each router sends out “threads”—short discovery packets—on all ports. Neighbors respond, forming local connectivity data structures.
	\item  	Local Weave: Threads intersect and overlap, letting routers learn about multi-hop neighbors.
	\item  	Damage Repair: If a link fails, local threads are resent to repair or reroute around the break.
	\item  Tension Metrics: Each link in the web holds a “tension” (latency, throughput) that can be monitored and used to shift traffic if congestion or errors rise.
\end{enumerate}

Benefits:
\begin{itemize}
	\item Resilience Through Redundancy: Overlapping “threads” ensure multiple known paths.
	\item Incremental Updates: Each node refines its local web structure.
	\item Ease of Local Addressing: Short IDs can be assigned to neighbors, aggregated as the web extends outward.
\end{itemize}

\end{document}