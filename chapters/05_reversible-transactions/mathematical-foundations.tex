\section{Mathematical Foundations}
\marginnote{Adapted from sections of ./AE-Specifications/chapters/@GPT-reversibility.tex}

In low-latency, high-throughput Layer 2 environments (e.g., Ethernet links), it's useful to model transactions as mathematical operations that can be precisely undone. This enables rollback, audit, and error recovery without heavyweight protocols.

\begin{enumerate}
\item \textbf{Model data as vectors.} \\
Each Ethernet frame is viewed as a vector in \(\mathbf{GF}(2)^n\), treating bits not as opaque payload but as elements in a vector space over a finite field.

\item \textbf{Transactions as invertible operations.} \\
The sender and receiver maintain a shared state \(S \in \mathbf{GF}(2)^n\). A transaction is an invertible linear transformation \(T\) applied to that state: \(S' = T(S)\). Because \(T\) is invertible, the original state can always be recovered via \(T^{-1}\).

\item \textbf{Reversibility via state updates.} \\
To reverse a transaction, one sends a message (or derivable signal) allowing the application of \(T^{-1}\). This guarantees deterministic rollback.
\end{enumerate}


We consider a chain of \(N+1\) nodes labeled \(A_0 \to A_1 \to \dots \to A_N\), where each node \(A_i\) maintains a local state vector \(S_i \in \mathbf{GF}(2)^n\), typically initialized to the all-zero vector \(0^n\) or some other agreed-upon state. Each link \((A_i \to A_{i+1})\) between adjacent nodes is associated with an invertible linear transformation \(T_{i,i+1}\), which governs how state updates propagate along the chain.


\subsection{Forward Execution}

To execute a transaction spanning all links:

\begin{enumerate}
\item At each hop \(i\), node \(A_i\) applies \(T_{i,i+1}\) to its state \(S_i\) and transmits the transformation to \(A_{i+1}\).
\item Node \(A_{i+1}\) applies the same \(T_{i,i+1}\) to its own state \(S_{i+1}\), maintaining link-local consistency.
\end{enumerate}

The result is a chained sequence of transformations:
\[
  S_i' = T_{i,i+1} \cdot S_i \quad \text{for } i = 0, 1, \dots, N-1,
\quad
S_N' = T_{N-1,N} \cdot S_N.
\]

\subsection{Rollback (Reverse Direction)}

Reversibility is achieved by applying the inverse transformations in reverse order:

\begin{enumerate}
\item Node \(A_N\) applies \(T_{N-1,N}^{-1}\) to revert \(S_N'\) to \(S_N\).
\item It signals node \(A_{N-1}\), which applies \(T_{N-1,N}^{-1}\) and then \(T_{N-2,N-1}^{-1}\), and so on.
\item This continues up the chain until \(A_0\) applies \(T_{0,1}^{-1}\), restoring the original \(S_0\).
\end{enumerate}

\subsection{Example: XOR-Based Masks}

If each \(T_{i,i+1}\) is a simple XOR with mask \(\Delta_{i,i+1}\), then:
\[
  S_i \mapsto S_i \oplus \Delta_{i,i+1}, \quad
  S_{i+1} \mapsto S_{i+1} \oplus \Delta_{i,i+1}.
\]
Reversing just involves reapplying the same mask due to \(\Delta \oplus \Delta = 0\).

\subsection{Notes on Synchronization}

\begin{itemize}
\item \textbf{Acknowledgments:} Each node should confirm that the next node has applied its transformation before committing its own.
\item \textbf{Composite View:} The full transaction across \(N\) links is a composition:
\[
  T_{\text{total}} = T_{N-1,N} \circ T_{N-2,N-1} \circ \dots \circ T_{0,1}.
\]
\item \textbf{Error Handling:} Any failure in transmission or transformation must be detected early, as desynchronization across nodes can compound. Redundant encodings, checksums, or commit/abort protocols may be used.
\end{itemize}


\section{Atomic Transactions on Æ-Link}

\subsection {One-Phase Commit}

\subsection {Two-Phase Commit}

\subsection {Four-Phase Commit}

\section{Flow Control and Backpressure}

\section{Transactions on Trees}