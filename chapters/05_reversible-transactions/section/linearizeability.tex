\documentclass[../../../OAE-SPEC-MAIN.tex]{subfiles}
\begin{document}

\section{Linearizability vs. Hypertransactions}
\marginnote{From DAE-Technical/Mulligan Stew.pages}

There is no such thing as a real-time order. Wall clocks are illusions; physics knows no absolute now.

\section{Linearizability and Its Limits}

\subsection{What is Linearizability?}
Linearizability is a consistency model for concurrent systems that provides a real-time ordering of operations, making the system's behavior appear as if operations were executed instantaneously at some point between their invocation and response.

\begin{description}
  \item[Real-Time Order:] Operations appear to take effect at a single point between invocation and response.
  \item[Consistency and Atomicity:] Guarantees transition between consistent states as if operations were sequential.
  \item[Indistinguishability:] Observers cannot distinguish between the actual execution and a hypothetical, linearized one.
\end{description}

\subsection{Physical Reality}
While this may be intuitive for computer scientists, modern physics rejects any notion of absolute real-time order. Special and General Relativity, as well as Quantum Mechanics, demonstrate that causality is indefinite, and entanglement is non-local.

\subsection{FITO Thinking}
Linearizability is a \textbf{Forward-In-Time-Only (FITO)} concept. It presumes monotonic progress, ignoring the need to reverse operations. In reality, reversing steps (like backing out of a one-way street) is essential to ensure availability and resilience.

\section{Hypertransactions}

\subsection{Reversible Subtransactions in FPGA}
Hypertransactions are \textit{reversible subtransactions executed entirely within the FPGA substructure}.

\begin{quote}
\textbf{Moss defines a transaction as:}
\textit{``collections of primitive actions that are indivisible, ensuring consistency even with concurrency or failure.''}
\end{quote}

Despite decades of effort, the industry has failed to achieve true indivisibility. Instead of enforcing an irreversible timeline, we introduce reversibility into the design.

\subsection{Nested Transactions and Synchronization}
Nested transactions enable nested universes of synchronization and recovery:
\begin{itemize}
  \item Subtransactions can fail independently without aborting the parent.
  \item Define ``reversibility points'' to backtrack from local failures.
\end{itemize}

We use `invocation` and `response` as reversible events to guide ancilla bits in managing forward/backward state machine evolution.

\subsection{From Locking to Multiway Systems}
Moss relies on locking, distributed commit, and deadlock detection. We move beyond this to MultiWay Systems, allowing multiple possible valid outcomes in superposition. Determinism is achieved \textit{on demand}.

\section{HyperIsolation: A New Atomicity Primitive}

\subsection{Atomicity Without Compromise}
DAE overcomes the performance/strictness tradeoff of conventional atomicity:
\begin{itemize}
  \item Violations become visible and recoverable.
  \item Cell agents monitor liveliness and reroute stalled links.
  \item Recovery occurs in 2–4 FPGA cycles (sub-microsecond), appearing as "unbreakable" to host processes.
\end{itemize}

\subsection{Reversible Definitions of Atomicity}
Where Moss defines atomicity in FITO terms, we extend it:
\begin{itemize}
  \item Proofs hold in both forward and reverse directions.
  \item Reverse path is mathematically modeled (e.g., invertible functions in linear algebra).
  \item Alternative rollback paths to safe states are enabled.
\end{itemize}

\section{Consistency Reimagined}

\subsection{Mathematical Consistency}
Our consistency primitive is rooted in formal mathematical completeness, compositionality, and simulation. Verified in Mathematica, compiled into Petri nets, and synthesized to Verilog.

\section{HyperIsolation: A New Isolation Primitive}
We introduce an entangled, two-phase commit which:
\begin{itemize}
  \item Transfers reversible tokens of varying size across FPGAs.
  \item Overcomes limitations of Serializability and S2PL.
  \item Prevents host-side blocking via TPI (Transaction Processing Interface).
\end{itemize}

\section{Physics and Networking: A Rebuttal to FITO}

\subsection{Real-Time Order is an Illusion}
Physics denies the concept of instantaneous global time. All observations are relative, and latency is inevitable.

\subsection{Consistency and Atomicity Are Observer-Relative}
\begin{itemize}
  \item Transitions appear different to each observer.
  \item Faults expose inconsistent sub-states that cannot be masked.
  \item Redundancy and availability metrics do not guarantee determinism.
\end{itemize}

Entanglement means maintaining consistent-but-complementary states between peers.

\subsection{Linearization Fails under Fault}
Packet drops, reordering, and duplication break coherence between senders and receivers. Indistinguishability is violated. Sequence numbers alone cannot recover lost structure.

\section{Sequence Numbers as Complex Modulo Counters}

\begin{itemize}
  \item Sequence numbers must be reversible modulo counters.
  \item Bounded interaction buffers (2, 4, ... N interactions) are provably atomic.
  \item We use complex-number representations, stored in lookup tables.
\end{itemize}

\textbf{Note:} Increasing counter width increases failure likelihood and recovery complexity.

\end{document}

