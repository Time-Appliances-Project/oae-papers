\documentclass[../../../OAE-SPEC-MAIN.tex]{subfiles}
\begin{document}

\section{Rethinking Atomicity: Counterfactual Transactions}
\marginnote{From ./AE-Specifications-ETH/standalone/Transactions-maybe-dup.tex}

This document challenges the Forward-In-Time-Only (FITO) assumptions behind conventional transactions in distributed systems. It argues that atomicity, as currently conceived, is a flawed abstraction and proposes a framework for reversible subtransactions as a more robust alternative.

\begin{quote}
\emph{“Transactions begin and they end.”}\\
\hspace*{0.5cm}---Charlie Johnson, TMF Product News
\end{quote}

This simple phrase conceals deep design hazards. Transactions appear to begin with a trigger and end with a commit, but in distributed systems, these bookends obscure severe internal inconsistencies. 

At issue are the mechanisms we use to track and guarantee these transactional intervals: timestamps, logs, filesystems, and even our concepts of causality. Each introduces cracks in the facade of atomicity.

\section{The Forward-In-Time-Only Fallacy}
\marginnote{FITO: Forward-In-Time-Only thinking assumes linear causality.}
Most distributed systems today adopt what we call \textbf{Forward-In-Time-Only (FITO)} thinking. That is:
\begin{enumerate}
  \item \textbf{Open a transaction} with a timestamp.
  \item \textbf{Apply a sequence} of operations.
  \item \textbf{Close the transaction} with a commit or rollback.
\end{enumerate}

But this approach breaks down under scrutiny.

\subsection*{Three FITO Hazards}
\begin{enumerate}
  \item \textbf{Timestamps are not unique.} Even on a single machine with GHz processors and nanosecond clocks, timestamp collisions occur. OS-level clock management does not guarantee uniqueness.
  \item \textbf{Timestamps are single points of failure.} Any drift, packet loss, or sync error in NTP/PTP introduces false ordering assumptions.
  \item \textbf{Simultaneity is an illusion.} Relativity tells us simultaneity is observer-dependent. Building global event orderings on timestamps is unsafe.
\end{enumerate}

\section{The False Comfort of Atomicity}
We often say: "all or nothing." But our stack is built on sand:
\begin{itemize}
  \item The database relies on the log.
  \item The log relies on the filesystem.
  \item The filesystem relies on \texttt{fsync}.
  \item \texttt{fsync} relies on storage hardware.
\end{itemize}
Each of these layers fails to guarantee true atomicity. When one fails, the recovery model becomes:\textbf{ Smash and Restart}.

\section{The Myth of Reliable Commit}
Protocols like Two-Phase Commit (2PC) attempt to enforce distributed agreement. But they depend on:
\begin{itemize}
  \item Log synchronization across nodes
  \item Network reliability
  \item Time-based coordination
\end{itemize}

When any assumption breaks, so does safety. Eventually, we replace consistency with survivability—and correctness with heuristics.

\section{Toward Reversible Thinking}
Suppose we reject FITO. Suppose we view the transaction as \emph{reversible}. 

\begin{quote}
If the forward protocol is correct, we can construct a reverse protocol.
\end{quote}

This leads to \textbf{reversible subtransactions}: bounded operations that can be undone without global rollback.

\subsection*{Counterfactual Transactions}
\begin{itemize}
  \item A transaction can end, then begin again.
  \item Logs become braids, not linear sequences.
  \item Atomicity becomes a constraint, not an assumption.
\end{itemize}

Inspired by Marletto's counterfactual physics, this model embraces partial reversibility as an engineering practice.

\section{Closing the Interval—Reopened}
\marginnote{Simultaneity is not fundamental. Causality is.}
"Closing the interval" with a commit only makes sense if we know the state is stable. In reality, it's a guess based on layers of non-atomic operations.

By rethinking transactions through reversible logic, we can:
\begin{itemize}
  \item Define precise causal dependencies
  \item Undo partial effects
  \item Recover without restart
\end{itemize}

Reversibility isn't science fiction. It is what rollback always wanted to be.

\section{Conclusion}
The abstraction of atomicity has outlived its usefulness as a guarantee. In modern distributed systems, FITO thinking and timestamp dependency introduce hazards we can no longer ignore.

It is time to engineer \textbf{reversible protocols}, built on causal semantics—not illusions of simultaneity. Let us design transactions that don't just commit or roll back, but that can \emph{unwind}.

\end{document}
