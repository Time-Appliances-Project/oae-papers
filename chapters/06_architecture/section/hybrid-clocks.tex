\documentclass[../../../OAE-SPEC-MAIN.tex]{subfiles}
\begin{document}


\section{Hybrid Clocks and Common Knowledge}

\marginnote{From ./AE-Specifications-ETH/standalone/hlc\_common\_knowledge.tex}


Hybrid clock research, led by S.\,S.\ Kulkarni and collaborators, welds
a physical‐time estimate to a Lamport‐style logical counter.
The result is a timestamp that (i) respects causality, (ii) stays within
a known skew~$\varepsilon$ of wall time, and (iii) avoids the commit‐wait
penalties of systems such as Google Spanner.
This handout summarizes the key papers, explains how Hybrid Logical
Clocks (HLC) operate, and analyzes their impact on the epistemic concept
of \emph{common knowledge}.

%--------------------------------------------------------------------
\begin{margintable}
\caption{Kulkarni hybrid‐clock lineage}
\begin{tabular}{@{\hspace{-25pt}}llp{1.25in}@{}}
\toprule
Year & Work & Key idea \\
\midrule
2012 & HybridTime          & Physical time $\oplus$ counter; bounds skew. \\
2014 & Hybrid Logical Clock & 64‑bit HLC preserves $e\!\to\!f\Rightarrow\mathrm{HLC}(e)<\mathrm{HLC}(f)$. \\
2015 & Hybrid Vector Clock  & Vector form; prunes entries older than $\varepsilon$. \\
16--20 & System integrations & HLC in GentleRain+, CausalSpartanX, NuKV. \\
\bottomrule
\end{tabular}
\end{margintable}

\subsection{How Hybrid Logical Clocks Work}

Each node maintains
\[
\text{HLC} = (\text{pt}, \text{ctr})
\]
where \texttt{pt} mirrors physical time and \texttt{ctr} counts logical
ties.

\begin{enumerate}[label=\arabic*.]
\item \textbf{Local event}\\
      \texttt{pt $\leftarrow$ now();}\\
      \texttt{pt $\leftarrow \max(\texttt{pt},\text{pt})$;}\\
      \texttt{ctr $\leftarrow$ (\texttt{pt==pt}) ? 0 : ctr+1;}
\item \textbf{Message receive $(m.\text{HLC})$}\\
      \texttt{pt $\leftarrow$ now();}\\
      \texttt{pt $\leftarrow \max(\texttt{pt},m.\text{pt})$;}\\
      \texttt{ctr $\leftarrow$ (pt==m.pt) ? $\max(\text{ctr},m.\text{ctr})+1$ : 0;}
\end{enumerate}

Guarantees:

\begin{itemize}
\item If event $e$ happens before $f$, then $\text{HLC}(e)<\text{HLC}(f)$.
\item Absolute error $|\text{pt}-\text{HLC}| \le \varepsilon$ as long as
      the underlying clock discipline meets that bound.
\end{itemize}

%--------------------------------------------------------------------
\subsection{Epistemic Implications}

\subsection*{Classical limit}
Halpern and Moses proved that in asynchronous systems absolute common
knowledge is impossible; only an infinite tower
$E\varphi,\,E\,E\varphi,\ldots$ is attainable.

\subsection*{Hybrid‐clock shift}
Because every timestamp deviates from real time by at most
$\varepsilon$, a node that observes $\text{HLC}\ge T$ can
\emph{deduce} that all events with timestamp $<T-\varepsilon$ are in its
past.  That turns the impossibility into a quantitative statement:
facts can become \emph{$\varepsilon$‐common knowledge}.

%\begin{marginfigure}
%\caption{Knowledge growth window of width~$\varepsilon$.}
%\includegraphics[width=\linewidth]{placeholder.pdf} % replace with real figure if desired
%\end{marginfigure}

\paragraph{Practical upshot}
Linearizable commits, causal session guarantees, and wait‐free
consistent snapshots become feasible after one round trip, provided the
skew bound holds.

\paragraph{Limitation}
If $\varepsilon$ blows up (GPS loss, NTP outliers, relativistic links)
the deduction fails and classical uncertainty returns. Hybrid clocks do
not escape Forward‑In‑Time‑Only thinking; they merely bound its error.

%--------------------------------------------------------------------
\section{Relation to FITO Perspective}

Your Forward‑In‑Time‑Only critique argues that Newtonian time is a
hidden axiom.  Hybrid clocks expose—rather than erase—this axiom by
making $\varepsilon$ explicit.  They therefore align with FITO analysis:
progress requires either

\begin{itemize}
\item shrinking $\varepsilon$ (better hardware sync), or
\item replacing deterministic bounds with probabilistic or
      reversible notions of order.
\end{itemize}

%--------------------------------------------------------------------
\section{Open Problems}

\begin{enumerate}
\item \textbf{Dynamic $\varepsilon$.}  Adapt clocks when skew drifts.
\item \textbf{Probabilistic knowledge.}  Treat timestamps as confidence
      intervals.
\item \textbf{ICO‑aware clocks.}  Design schemes that tolerate
      indefinite causal order and reversible transactions.
\item \textbf{Eventual common knowledge.}  Combine HVC pruning with DAG
      gossip in partitioned networks.
\end{enumerate}

%--------------------------------------------------------------------
\section{Takeaways}

Hybrid clocks bridge logical causality and imperfect wall time,
achieving the effect of common knowledge after a bounded delay
$\varepsilon$.  They power modern geo‐replicated stores without heavy
coordination cost, but remain inside the FITO worldview.  Future work
will loosen or replace the global arrow of time.

%--------------------------------------------------------------------
\section*{References}

\begin{enumerate}[label={[}\arabic*{]}]
\item S.\,S.\ Kulkarni and N.\ Mittal.  HybridTime: A decoupling of
      coordination and time in distributed systems.  TR, 2012.
\item S.\,S.\ Kulkarni, et al.  Logical Physical Clocks and Consistent
      Snapshots.  2014.
\item V.\ Karmarkar and S.\,S.\ Kulkarni.  Bounds on Hybrid Vector
      Clocks.  IEEE SRDS, 2015.
\item H.\ Wu, et al.  CausalSpartanX: Causal consistency over HLC.
      Middleware, 2016.
\item J.\ Su, et al.  NuKV: Building a scalable and reliable KV store.
      SoCC, 2020.
\item J.\,Y.\ Halpern and Y.\ Moses.  Knowledge and common knowledge in
      distributed environments.  JACM 37(3), 1990.
\end{enumerate}

\end{document}
