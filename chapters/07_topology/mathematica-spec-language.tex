\section{Mathematica as a Specification Language}
\marginnote{From ./AE-Specifications-ETH/standalone/Mathematica-Spec-Language.tex}

Exploring \textbf{formally executable specifications} in \textbf{datacenter architecture} touches the core of verifiability, reproducibility, and automation in modern systems.

\subsection*{Definition: Formally Executable Specification}

\begin{marginfigure}
\small
A \emph{formally executable specification} is:
\begin{itemize}[leftmargin=*]
  \item \textbf{Precise and unambiguous:} defined mathematically or via formal syntax.
  \item \textbf{Executable:} interpretable or simulatable.
  \item \textbf{Deterministically testable:} consistent output for consistent input.
\end{itemize}
\end{marginfigure}

In datacenter contexts, this implies that hardware, networking, storage, and compute orchestration policies are:
\begin{itemize}
  \item Executable in simulation or emulation environments,
  \item Amenable to formal verification for correctness, safety, and performance.
\end{itemize}

\subsection*{Why It Matters in Datacenters}

\begin{itemize}
  \item \textbf{Correctness:} validate failover, routing, and policy enforcement.
  \item \textbf{Optimization:} evaluate configurations automatically.
  \item \textbf{Security:} prove isolation and policy compliance.
  \item \textbf{Confidence:} ensure safe deployment at scale.
\end{itemize}

\subsection*{Relevant Tools and Technologies}

\begin{marginfigure}
\footnotesize
\begin{tabularx}{\linewidth}{@{}lX@{}}
\toprule
\textbf{Domain} & \textbf{Tools} \\
\midrule
Network Architecture & P4, TLA+, NetKAT, Batfish \\
Storage Systems & TLA+, Ivy, Alloy, Z3 SMT \\
Orchestration & Kubernetes CRDs, Pulumi, OPA, Nomad \\
Formal Languages & TLA+, Coq, Lean, Dafny, Alloy \\
Execution & Mininet, NS-3, OMNeT++, QEMU, Verilator \\
\bottomrule
\end{tabularx}
\caption{Selected tools for formally modeling datacenter systems}
\end{marginfigure}

\subsection*{Example: Rack-Aware Topology Specification}

Imagine a model with:
\begin{itemize}
  \item Compute nodes linked via ToR (Top-of-Rack) switches,
  \item Spine switches in a leaf-spine topology,
  \item Multi-path routing and QoS,
  \item VM placement and replication constraints.
\end{itemize}

The spec could:
\begin{itemize}
  \item Simulate failures and load distribution,
  \item Detect routing loops or black holes,
  \item Evaluate bandwidth and latency guarantees,
  \item Prove placement constraints meet SLAs.
\end{itemize}

\subsection*{Vision: ``Datacenter-as-Code'' Verified}

\begin{itemize}
  \item High-level specs compile into deployable artifacts,
  \item Every change is property-checked and testable,
  \item Infrastructure becomes version-controlled logic, replacing spreadsheets and tribal lore.
\end{itemize}

\subsection*{Evaluating Mathematica for Executable Specification}

Mathematica is a powerful computational platform. Its value depends on whether expressiveness or formal rigor is the priority.

\subsection{Strengths of Mathematica}

\begin{marginfigure}[+80mm]
\scriptsize
\begin{tabularx}{\linewidth}{@{}lX@{}}
\toprule
\textbf{Category} & \textbf{Capability} \\
\midrule
Symbolic Computation & Excellent for pipelines, graphs, latency models \\
Executability & Immediate execution and visualization \\
Expressiveness & Supports discrete, continuous, algebraic models \\
Rapid Prototyping & Rich in units, semantics, interactivity \\
Logic Tools & First-order logic, SAT solving, quantifiers \\
Documentation & Notebooks are self-contained and reproducible \\
\bottomrule
\end{tabularx}
\caption{Strengths of Mathematica in system modeling}
\end{marginfigure}

\subsection*{Limitations Compared to Formal Languages}

\begin{itemize}
  \item \textbf{Formal Semantics:} lacks type theory foundations (Coq, Lean).
  \item \textbf{Verification:} no native model checking or invariant proofs.
  \item \textbf{Concurrency:} no Lamport clocks or message-passing models.
  \item \textbf{Determinism:} pattern matching may be nondeterministic.
  \item \textbf{Refinement:} lacks formal spec-to-implementation pathways.
\end{itemize}

\subsection*{Suitable Use Cases}

\begin{itemize}
  \item Modeling tradeoffs in resource allocation,
  \item Simulating flows using graph theory,
  \item Prototyping performance constraints,
  \item Symbolic scheduling and placement logic,
  \item Writing executable whitepapers with computation and code.
\end{itemize}

\subsection*{Where It Falls Short}

\begin{itemize}
  \item Verifying safety and liveness across all states,
  \item Proving conformance or refinement,
  \item Modeling concurrency and faults rigorously,
  \item Integrating with RTL verification pipelines,
  \item Participating in formal proof communities.
\end{itemize}

\subsection{Summary Judgment}

\begin{quote}
Mathematica is:
\begin{itemize}
  \item \textbf{Excellent} for exploratory, high-level modeling and simulation,
  \item \textbf{Weak} for formal verification, proofs, and correctness guarantees,
  \item \textbf{Valuable} as a literate architecture spec tool, but not a full formal methods platform.
\end{itemize}
\end{quote}

\subsection{Appendix A: TLA+ Model -- Rack-Aware Topology}

\begin{marginfigure}
\footnotesize
\textbf{RackAwareSpec.tla}
\end{marginfigure}

\begin{quote}
\scriptsize
\begin{verbatim}
------------------------------ MODULE RackAwareSpec ------------------------------
EXTENDS Naturals, Sequences

CONSTANTS Racks, Nodes, Links

VARIABLES rackStatus, linkStatus, trafficMap

(*--algorithm RackAware
variables rackStatus \in [Racks -> {"up", "down"}],
          linkStatus \in [Links -> {"up", "down"}],
          trafficMap \in [Nodes -> [Nodes -> {"ok", "blocked", "reroute"}]];

define
  IsAvailable(n) == \E r \in Racks: rackStatus[r] = "up" /\ n \in Nodes /\ TRUE
end define;

begin
  Init ==
    /\ \A r \in Racks: rackStatus[r] = "up"
    /\ \A l \in Links: linkStatus[l] = "up"
    /\ \A s, d \in Nodes: trafficMap[s][d] = "ok";

  Next ==
    \E r \in Racks:
      /\ rackStatus[r] = "up"
      /\ rackStatus' = [rackStatus EXCEPT ![r] = "down"]
      /\ UNCHANGED <<linkStatus, trafficMap>>
  \/ \E l \in Links:
      /\ linkStatus[l] = "up"
      /\ linkStatus' = [linkStatus EXCEPT ![l] = "down"]
      /\ UNCHANGED <<rackStatus, trafficMap>>;

end algorithm;
===============================================================================
\end{verbatim}
\end{quote}

\newpage
\section*{Appendix B: Alloy Model -- Storage Placement Constraints}

\begin{marginfigure}
\footnotesize
\textbf{StorageModel.als}
\end{marginfigure}

\begin{quote}
\scriptsize
\begin{verbatim}
module StorageModel

abstract sig Rack {}
sig Node {
  hostRack: one Rack,
  stores: set Volume
}
sig Volume {
  replicas: some Node
}

fact ReplicationFactor {
  all v: Volume | #v.replicas = 3
}

fact NoReplicaOnSameRack {
  all v: Volume |
    all disj n1, n2: v.replicas |
      n1.hostRack != n2.hostRack
}

pred ShowExample {}

run ShowExample for 3 Rack, 6 Node, 2 Volume
\end{verbatim}
\end{quote}
