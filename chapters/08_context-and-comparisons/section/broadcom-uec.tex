\documentclass[../../../OAE-SPEC-MAIN.tex]{subfiles}
\begin{document}

\section{Ultra Ethernet}
% \marginnote{From ./AE-Specifications-ETH/standalone/Broadcom-UEC}

% ---------------------------------------------------------------------------
\subsection{Executive overview}

Ultra Ethernet Consortium (UEC) is producing an \textbf{open, Ethernet-based full-stack
specification} targeted at AI and HPC fabrics with one million endpoints,
terabit links and sub-microsecond tail latency.
Its clean-slate \emph{Ultra Ethernet Transport (UET)} replaces legacy RoCE,
adds packet-spray multipath, flexible ordering, congestion-adaptive spraying
and hardware collectives.

Compared with InfiniBand, UET keeps the broad Ethernet ecosystem while
approaching supercomputer-interconnect performance.

% ---------------------------------------------------------------------------
\subsection{Scope and objectives}

\begin{itemize}
  \item \textbf{Scale:} up to 1e6 endpoints and 800;1600 Gbps links.
  \item \textbf{Latency target:} keep 99.9\,\% of collectives within one-digit
        \si{\micro\second}
  \item \textbf{Profiles:} \emph{AI Base}, \emph{AI Full}, and \emph{HPC}; same wire
        protocol, differing verb sets.
  \item \textbf{Layer coverage:} Physical, Link, Transport, Software API,
        Security, Telemetry/Management.
\end{itemize}

% ---------------------------------------------------------------------------
\subsection{Architecture at a glance}

\subsection*{Transport layer – UET}
\begin{itemize}
  \item \textbf{Ephemeral connections}: no handshake; state cached and
        reclaimed per transaction (memory-efficient at scale).
  \item \textbf{Packet spraying} across all equal-cost paths with per-packet
        sequence numbers; ordering enforced on message completion only.
  \item \textbf{Reliability} uses selective retry plus optional
        \emph{packet trimming} (header delivered, payload discarded)
        to signal incipient congestion without head-of-line
        blocking.
\end{itemize}

\subsection*{Congestion control}
Two complementary schemes:  
\begin{enumerate}[label={(\arabic*)},leftmargin=*]
  \item \textbf{Sender-adaptive AI/MD} window (\textasciitilde{}RTT-speed
        reaction, ECN-driven).  
  \item \textbf{Credit-grant} receiver pacing for extreme
        incast.
\end{enumerate}

\subsection*{Link layer}
\begin{itemize}
  \item \textbf{Link-Layer Retry (LLR)} negotiated via extended LLDP; hop-by-hop
        NACK keeps BER-induced loss from bubbling up.
  \item \textbf{Two traffic classes} prevent deadlock and prioritise control
        traffic even on PFC-free fabrics.
\end{itemize}

\subsection*{Software API}
Based on \texttt{libfabric 2.0} with new verbs:
\emph{Deferrable Send}, optimistic rendezvous, expanded atomics and tagged
collectives.

\subsection*{Security}
Group-keyed AES-GCM; leverages IPSec/PSP primitives yet preserves
ephemeral-connection model—important for accelerator offload.

\subsection*{In-network collectives (INC)}
Lightweight header extensions allow switches to execute
\textit{reduce}, \textit{broadcast} and \textit{all-reduce} without CPU
round-trips, standardising what was previously vendor-proprietary.

% ---------------------------------------------------------------------------
\subsection{Key innovations vs.\ legacy RDMA}

\begin{table}[ht]
\footnotesize
\caption{Ultra Ethernet vs.\ RoCEv2 / InfiniBand RC.}
\begin{tabular}{@{}p{0.25\textwidth}p{0.32\textwidth}p{0.32\textwidth}@{}}
\toprule
\textbf{Feature} & \textbf{Ultra Ethernet} & \textbf{RoCEv2 / IB RC}\\\midrule
Multipath & Packet-level spray & Flow-hash ECMP\\
Ordering & Flexible, message-level & Strict in-order\\
Loss recovery & Selective, trim-aware & Go-back-N\\
Congestion ctrl. & Auto, zero-tune & DCQCN (manual tune)\\
Connection state & Ephemeral, pooled & Per-peer, long-lived\\
Collectives & INC standard & Vendor proprietary / host SW\\
Security & Built-in AES-GCM & External overlay\\
\bottomrule
\end{tabular}
\end{table}

% ---------------------------------------------------------------------------
\subsection{Performance and scalability}

UEC’s internal simulations on Clos-4 fabrics show $>$95\,\% link utilisation
and $<$5\,\si{\micro\second} 99.9-percentile all-reduce latency on 4096-GPU
clusters at \SI{800}{Gbps}.
Early silicon demos (Q1 2025) from Broadcom’s
\texttt{BCM57608} NIC confirm wire-rate UET on existing
PAM4 links.

% ---------------------------------------------------------------------------
\subsection{Ecosystem readiness}

\begin{itemize}
  \item \textbf{Membership}: 90+ vendors incl.\ AMD, Arista, Broadcom,
        Cisco, Google, Meta, Microsoft, Intel.
  \item \textbf{Reference code}: open-source UET over commodity NICs,
        easing migration from libfabric/RDMA apps.
  \item \textbf{Timeline}: v1.0 spec ratification Q3 2025; first commercial
        gear shipping same quarter.
\end{itemize}

% ---------------------------------------------------------------------------
\subsection{Gaps and open issues}
\begin{itemize}
  \item \textbf{Standardisation overlap} with IEEE 802.1Q priorities,
        IETF congestion-control drafts and emerging UALink fabric.
  \item \textbf{Inter-op testing} required for packet trimming and INC
        header parsing across multiple switch ASIC generations.
  \item \textbf{Management model}: YANG/Redfish bindings still draft.
\end{itemize}

% ---------------------------------------------------------------------------
\subsection{FITO analysis}
Conventional Ethernet handles failure strictly forward-in-time:
lost packets are re-sent, global ordering re-asserted \emph{after}
the fact. UET’s \emph{selective trimming} preserves the causal
header even under congestion; the payload can be retro-materialised
without rolling back link-layer state—an incremental step toward
reversible, idempotent communication you favour in
\textsc{Open Atomic Ethernet} research.

% ---------------------------------------------------------------------------
\subsection{Conclusion}
Ultra Ethernet modernises RDMA semantics for AI/HPC while preserving
the economic and tooling advantages of Ethernet.  
If the consortium delivers on its v1.0 schedule and multi-vendor
interoperability, UET could displace proprietary HPC fabrics and close
the performance gap with InfiniBand—especially when paired with
Broadcom’s Scale-Out Ethernet implementation.

% ---------------------------------------------------------------------------
%\begin{thebibliography}{9}\footnotesize
%\bibitem{UECOverview2023} Ultra Ethernet Consortium,
%\emph{UEC 1.0 Overview and Motivation},
%July 2023.  % :contentReference[oaicite:17]{index=17}
%\bibitem{UECSpecUpdate2024} Ultra Ethernet Consortium,
%``Ultra Ethernet Specification Update,'' Blog post,
%Aug 2024.  % :contentReference[oaicite:18]{index=18}
%\bibitem{UECPresentation2023} Ultra Ethernet Consortium,
%\emph{Consortium Overview Presentation}, Aug 2023.  % :contentReference[oaicite:19]{index=19}
%\bibitem{Hoefler2023} T.~Hoefler \textit{et al.},
%``Data-Center Ethernet and RDMA: Issues at Hyperscale,''
%\emph{IEEE Computer}, July 2023.
%\bibitem{CloudSwitchBlog2024} CloudSwit.ch,
%``In-depth Exploration of UEC,'' Blog post, Oct 2024.  % :contentReference[oaicite:20]{index=20}
%\bibitem{SNIA2024} J.~Metz,
%\emph{Ethernet Evolved: AI’s Future with UEC},
%SNIA CMS Summit, 2024.  % :contentReference[oaicite:21]{index=21}
%\end{thebibliography}

\end{document}
