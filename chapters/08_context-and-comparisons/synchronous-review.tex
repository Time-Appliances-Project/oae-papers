\section{Review of ``Synchronous, Asynchronous and Causally Ordered Communication''}
\marginnote{From ./Atomic-Ethernet/AE-Specifications-ETH/standalone/syncnronous-review.tex}


%--------------------------------------------------------------------
\subsection{Why this paper still matters}
\begin{itemize}
  \item Unifies three delivery disciplines—synchronous, FIFO, and causal—under one axiomatic roof.
  \item Establishes the strict inclusion chain
        \[
          \text{RSC} \subset \text{CO} \subset \text{FIFO} \subset \text{A},
        \]
        clarifying what extra guarantees are purchased (and at what cost).
  \item Introduces the \emph{crown criterion}: a linear‑time test to decide whether an execution can be replayed with rendezvous semantics.
  \item Demonstrates that classic control algorithms (e.g.\ Dijkstra–Feijen–van Gasteren termination detection) remain safe under the weaker CO model.
\end{itemize}

%--------------------------------------------------------------------
\subsection{Core concepts and results}

\subsection{Model vocabulary}
\begin{description}[leftmargin=*,style=nextline]
  \item[A‑computation] Fully asynchronous; Lamport’s three happens‑before axioms.
  \item[FIFO‑computation] Adds per‑channel ordering (send order $\Rightarrow$ receive order).
  \item[CO‑computation] Globalises FIFO: if $\text{send}_1 \prec \text{send}_2$ (causally), then $\text{recv}_1 \prec \text{recv}_2$.
  \item[RSC‑computation] There exists a \emph{non‑separated linear extension} where every send is immediately followed by its receive.
  \item[Crown] Alternating sequence $\langle s_0,r_0,s_1,r_1,\dots,s_k,r_k\rangle$ forming a dependency cycle; its presence \emph{precludes} synchronous realisation.
\end{description}

\subsection{Hierarchy (all containments strict)}
\[
  \boxed{\text{RSC}} \;\subset\;
  \boxed{\text{CO}} \;\subset\;
  \boxed{\text{FIFO}} \;\subset\;
  \boxed{\text{A}}
\]

\subsection{Termination‑detection case study}
CO suffices for the Dijkstra ring‑token detector because any basic message can cross \emph{at most one} wave; the colouring rule then guarantees safety.

%--------------------------------------------------------------------
\subsection{Implementation guidance (1992 vintage)}
\begin{itemize}
  \item \textbf{FIFO}: Per‑link sequence numbers plus buffering.
  \item \textbf{CO}: Vector (or matrix) clocks, or handshake I/O buffers that forbid indirect overtakes.
  \item \textbf{RSC}: FIFO + per‑message acknowledgement that blocks the sender (classic rendezvous).
\end{itemize}

%--------------------------------------------------------------------
\subsection{Strengths and limitations}

\subsection*{Strengths}
\begin{itemize}
  \item Fully axiomatic—no reliance on wall‑clock bounds.
  \item Crown test is linear‑time and graph‑theoretic.
  \item Bridges theory and practice with concrete protocols.
\end{itemize}

\subsection*{Limitations}
\begin{itemize}
  \item Assumes reliable point‑to‑point channels (no loss, duplication, or Byzantine faults).
  \item Vector/matrix metadata scales poorly for thousands of nodes; modern systems use compaction.
\end{itemize}

%--------------------------------------------------------------------
\subsection{FITO perspective}
\marginnote{\footnotesize\raggedright%
\textbf{FITO lens:}\;Every step up the hierarchy embeds more irreversible forward‑time coupling.}
\begin{enumerate}[label=\arabic*.]
  \item \textbf{A}: progress driven by timeouts and retries—quintessential Forward‑In‑Time‑Only thinking.
  \item \textbf{CO}: removes physical‑time dependence; ordering relies only on causal DAG, admitting limited reversibility of buffering.
  \item \textbf{RSC}: rendezvous collapses send/receive into a single spacetime point, eliminating alternative orders.
\end{enumerate}

%--------------------------------------------------------------------
\subsection{Comparative note}
\begin{description}[leftmargin=*,style=nextline]
  \item[Lamport (1978)] Happened‑before for totally asynchronous systems.
  \item[Birman \& Joseph (1987)] Causal broadcast within groups; present paper generalises to point‑to‑point and situates it in the hierarchy.
  \item[Fidge (1991)] Vector clocks; adopted here as one implementation route.
\end{description}

%--------------------------------------------------------------------
\subsection{Key takeaways}
\begin{enumerate}
  \item Ordering guarantees form a strict lattice; algorithm soundness depends on picking the right rung.
  \item “Crown‑free $\iff$ synchronisable”: if no crowns exist, rendezvous replay is possible.
  \item Many rendezvous‑based algorithms work under CO, avoiding blocking latency.
  \item FIFO alone is \emph{insufficient} for algorithms that assume “no messages in flight.”
  \item Implementation cost rises sharply: RSC $\to$ latency; CO $\to$ metadata; FIFO $\to$ buffer space.
\end{enumerate}

%--------------------------------------------------------------------
%\subsection*{Bibliographic reference}
%B.\ Charron‑Bost, F.\ Mattern, and G.\ Tel.\\
%“Synchronous, Asynchronous, and Causally Ordered Communication.”\\
%\emph{Distributed Computing} 6 (1992): 173–191.
