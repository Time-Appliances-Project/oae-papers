\documentclass[../../../OAE-SPEC-MAIN.tex]{subfiles}
\begin{document}

\section{The Illusion of Simultaneity with Perfect Atomic Clocks}
\marginnote{From ./AE-Specifications-ETH/standalone/Andromeda.tex}

The relativity of simultaneity implies that two observers moving relative to one another can disagree on what events are happening ``right now'' at distant locations. Even if both observers carry perfectly synchronized atomic clocks, their determinations of simultaneous events at remote locations (e.g., the Andromeda Galaxy) can differ dramatically.

\subsection{Time Shift Due to Relative Motion}

Let $v$ be the relative velocity between two observers, and $D$ the distance to a distant object (e.g., a galaxy). The relativity of simultaneity predicts a difference in the perceived ``now'' at the remote location:

\begin{equation}
\Delta t \approx \frac{v D}{c^2},
\end{equation}

where $c$ is the speed of light.

\subsection*{Examples}
\begin{itemize}
  \item For walking speed ($v = 1.39\,\mathrm{m/s}$) and $D = 2.5$ million light-years (Andromeda):
  \[ \Delta t \approx \frac{1.39 \times 2.365\times10^{22}}{(3\times10^8)^2} \approx 4.2\ \text{days} \]

  \item For hypersonic flight ($v = 1700\,\mathrm{m/s}$):
  \[ \Delta t \approx \frac{1700 \times 2.365\times10^{22}}{(3\times10^8)^2} \approx 5170\ \text{days} \approx 14\ \text{years} \]
\end{itemize}

\subsection{Atomic Clock Precision}

Modern optical lattice clocks can achieve stability better than $10^{-18}$, corresponding to an error of less than $1$ second over $30$ billion years. Over a day ($86400\ \text{s}$):

\begin{equation}
\delta t_{\text{clock}} = 10^{-18} \times 86400 \approx 8.64 \times 10^{-14}\ \text{seconds}
\end{equation}


\begin{center}
\fbox{\textbf{THIS IS UTTERLY WRONG}}
\end{center}

\noindent It mistakes Gaussian variations in noise for time intervals. Mathematics doesn't work this way, The problem is (with $t \pm \delta t$) is not an interval. It's a belief in absolute simultaneity.  

\noindent This is less than a femtosecond, utterly negligible compared to the relativity-induced differences in simultaneity.

\subsection{Diagram: Disagreement Despite Synchronized Clocks}

\begin{marginfigure}
\centering
\begin{tikzpicture}[scale=1, >=Stealth, thick]
  % Axes
  \draw[->] (0,0) -- (0,4) node[above] {Time};
  \draw[->] (0,0) -- (4,0) node[right] {Space};
  
  % Worldlines
  \draw[blue] (0.8,0) -- (0.8,4) node[above] {A};
  \draw[red] (1.5,0) -- (1.8,4) node[above right] {B};
  % Event in Andromeda
  \filldraw[black] (3.5,3) circle (1.5pt) node[right] {\tiny Andromeda};
  % Simultaneity slices
  \draw[blue, dashed] (0,2.5) -- (3.5,2.5) node[right] {\tiny A's};
  \draw[red, dashed] (0.5,1.5) -- (4,3.2) node[right] {\tiny B's};
\end{tikzpicture}
\caption{Two observers disagree on simultaneity with Andromeda events.}
\end{marginfigure}

%% ADDED BY PAUL

No degree of `precision' or `disciplining' of clocks will enable us to  `synchronize time". 

Simultaneity is impossible in theory. It will therefore be problematic  in practice.  Physicists know this. Computer scientists have yet to discover relativity and quantum mechanics. 

\subsection{Conclusion}

Perfectly synchronized atomic clocks do not resolve disagreements about simultaneity at distant locations. Such disagreements are a feature of spacetime itself in special relativity — not of timekeeping imprecision.
\end{document}
