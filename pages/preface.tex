\documentclass[../OAE-SPEC-MAIN.tex]{subfiles}


\begin{document}

\vspace{150mm}


\vspace*{0.25\textheight}

\begin{fullwidth}
\begin{center}
    %{\Huge\bfseries Parsimony in Protocol Design}
    
    %\vspace{1.5em}
    
    \begin{minipage}{0.8\textwidth}
        \centering
        \emph{“Perfection is achieved not when there is nothing more to add,\\
        but when there is nothing left to take away.”}\\
        \vspace{0.5em}
        — \textsc{Antoine de Saint-Exupéry}
    \end{minipage}

\vspace{3em}

\noindent
\begin{minipage}{4.5in} % ~1in margins on 8.5in paper
\setlength{\parskip}{0.5em}

The primary goal of \AE thernet is to re-examine the foundational assumptions made by Ethernet five decades ago, and ask whether they remain valid within today’s racks and chiplet modules.

\vspace{1.5em}

Just as RISC processors transformed computing by embracing simplicity and focus, so too must we consider whether a ``reduced'' protocol can lead to a network that is formally verifiable, inherently transactional, and radically simpler.

\vspace{1.5em}

We adopt the same spirit as the original Ethernet pioneers: start from first principles, strip down to essentials, and then --- only then --- ask how best to interoperate with the world we inherit.

\end{minipage}
\end{center}
\end{fullwidth}

\end{document}