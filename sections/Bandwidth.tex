\documentclass[../HFT-main.tex]{subfiles} % NEW VERSION IS CASE.tex
\begin{document}
\section*{Ethernet Channel Capacity and Utilization Model}

\subsection*{Parameters and Assumptions}

\begin{itemize}
    \item Minimum Ethernet frame size: 64 bytes (512 bits)
    \item Interpacket Gap (IPG): 12 bytes (96 bits)
    \item Preamble + SFD: 8 bytes (64 bits)
    \item Total frame size (including overhead): $64 + 8 + 12 = 84$ bytes (672 bits)
    \item Propagation speed in cable: approximately $2 \times 10^{8}$ m/s (5 ns per meter)
    \item Fixed delays (clock jitter, transistor, BGA/PCB traces, SFP+): $\approx 1.2$ ns one-way
\end{itemize}

\subsection*{Transmission Time per Frame}
\[
T_{tx} = \frac{672\,\text{bits}}{\text{Link Rate}}
\]
\begin{tabular}{@{}lll@{}}
\toprule
Speed & Transmission Time & Calculation\\
\midrule
100 Gbps & 6.72 ns & $672/100\times10^{9}$ \\
10 Gbps & 67.2 ns & $672/10\times10^{9}$ \\
\bottomrule
\end{tabular}

\subsection*{Propagation Delays}
\begin{tabular}{@{}lc@{}}
\toprule
Cable Length & Propagation Delay (one-way) \\
\midrule
10 m & 50 ns \\
1 m & 5 ns \\
10 cm & 0.5 ns \\
1 cm & 0.05 ns \\
\bottomrule
\end{tabular}

\subsection*{Round Trip Time (RTT)}
\[
\text{RTT} = 2 \times (\text{Propagation Delay} + \text{Fixed Delay})
\]
\begin{tabular}{@{}lccc@{}}
\toprule
Length & Propagation & Fixed Delay & RTT \\
\midrule
10 m & 50 ns & 1.2 ns & 102.4 ns \\
1 m & 5 ns & 1.2 ns & 12.4 ns \\
10 cm & 0.5 ns & 1.2 ns & 3.4 ns \\
1 cm & 0.05 ns & 1.2 ns & 2.5 ns \\
\bottomrule
\end{tabular}

\subsection*{Effective Channel Utilization}
Assuming one ACK per packet, the cycle time per packet is:
\[
T_{cycle} = T_{tx} + \text{RTT}
\]
Utilization is then:
\[
U = \frac{T_{tx}}{T_{cycle}}
\]

\paragraph{Utilization at 100 Gbps}
\begin{tabular}{@{}lccc@{}}
\toprule
Length & $T_{tx}$ & Cycle Time & Utilization \\
\midrule
10 m & 6.72 ns & 109.12 ns & 6.16\% \\
1 m & 6.72 ns & 19.12 ns & 35.14\% \\
10 cm & 6.72 ns & 10.12 ns & 66.40\% \\
1 cm & 6.72 ns & 9.22 ns & 72.88\% \\
\bottomrule
\end{tabular}

\paragraph{Utilization at 10 Gbps}
\begin{tabular}{@{}lccc@{}}
\toprule
Length & $T_{tx}$ & Cycle Time & Utilization \\
\midrule
10 m & 67.2 ns & 169.6 ns & 39.62\% \\
1 m & 67.2 ns & 79.6 ns & 84.42\% \\
10 cm & 67.2 ns & 70.6 ns & 95.18\% \\
1 cm & 67.2 ns & 69.7 ns & 96.41\% \\
\bottomrule
\end{tabular}

\subsection*{Conclusions}
Shorter cable lengths greatly enhance utilization, especially critical at higher speeds. RTT significantly impacts channel efficiency at high data rates, making ultra-short connections advantageous.

\appendix
\section{Utilization for 1500 Byte and 9000 Byte Jumbo Frames}

\paragraph{Transmission Times}
For frames including overhead:
\begin{itemize}
    \item 1500-byte frame: 1520 bytes = 12160 bits
    \item 9000-byte jumbo frame: 9020 bytes = 72160 bits
\end{itemize}

\begin{tabular}{@{}lcc@{}}
\toprule
Speed & 1500-byte Frame & 9000-byte Frame \\
\midrule
100 Gbps & 121.6 ns & 721.6 ns \\
10 Gbps & 1.216 µs & 7.216 µs \\
\bottomrule
\end{tabular}

\paragraph{Estimated Utilizations}

\begin{tabular}{@{}lcccc@{}}
\toprule
Speed & Cable Length & Frame Size & Cycle Time & Utilization (\%) \\
\midrule
100 Gbps & 10 m & 1500 bytes & 224 ns & 54.29\% \\
100 Gbps & 10 m & 9000 bytes & 824 ns & 87.57\% \\
10 Gbps & 10 m & 1500 bytes & 1.318 µs & 92.23\% \\
10 Gbps & 10 m & 9000 bytes & 7.318 µs & 98.60\% \\
100 Gbps & 1 m & 1500 bytes & 134 ns & 90.75\% \\
100 Gbps & 1 m & 9000 bytes & 734 ns & 98.31\% \\
100 Gbps & 50 cm & 1500 bytes & 129.6 ns & 93.83\% \\
100 Gbps & 50 cm & 9000 bytes & 729.6 ns & 98.90\% \\
100 Gbps & 10 cm & 1500 bytes & 125.6 ns & 96.82\% \\
100 Gbps & 10 cm & 9000 bytes & 725.6 ns & 99.45\% \\
100 Gbps & 1 cm & 1500 bytes & 124.1 ns & 97.99\% \\
100 Gbps & 1 cm & 9000 bytes & 724.1 ns & 99.65\% \\
\bottomrule
\end{tabular}

Larger frames significantly improve utilization by reducing the relative overhead of RTT.


\section{Previous attempt.  Lets see what got lost?}

\section*{Ethernet Analysis}

This document provides a comprehensive outline for modeling channel capacity and utilization tailored specifically to Ethernet environments with varying cable lengths (10m, 1m, 10cm, and 1cm). Delays from an on-chip clock source are carefully included for detailed analysis.

\section{Model Parameters and Assumptions}

We assume standard Ethernet conditions (10Gbps and 100Gbps Ethernet) over copper cables, typical references being 10GBASE-T or 100GBASE-CR4.

\paragraph{Key assumptions:}
\begin{itemize}
	\item Data rate: Typically 10Gbps or 100Gbps.
	\item Propagation velocity in cable: $2 \times 10^8$ m/s.
	\item Serialization/Deserialization (SerDes) latency.
	\item MAC/PHY processing delays.
	\item On-chip clock source jitter and stability.
\end{itemize}

\section{Calculating Propagation Delays}

Propagation delay $t_p$ for cable length $L$ is:
\[
t_p = \frac{L}{v},\quad v = 2 \times 10^8\,\text{m/s}
\]

\begin{tabular}{@{}lcccc@{}}
\toprule
Length (m) & 10 & 1 & 0.1 & 0.01\\
Propagation Delay (ns) & 50 & 5 & 0.5 & 0.05 \\
\bottomrule
\end{tabular}

\section{On-Chip Clock Source Delays and Jitter}
Typical internal PLL or oscillator delays:
\begin{itemize}
	\item Clock distribution latency: 0.1–1 ns
	\item PLL Lock and Stability Jitter: 1–10 ps RMS
	\item Overall timing uncertainty (setup + hold margins): ~0.2–1 ns
\end{itemize}
We estimate total on-chip latency at ~1 ns.

\section{SerDes Latency}
Typical SerDes latency at 10–100Gbps is approximately 30 ns total (TX + RX).

\section{Total Latency per Cable Length}

Total Latency $T_{total}$ is:
\[
T_{total} = t_{\text{on-chip}} + t_{\text{SerDes}} + t_{\text{propagation}}
\]

\begin{tabular}{@{}ccccc@{}}
\toprule
Length (m) & $t_{\text{on-chip}}$ (ns) & $t_{\text{SerDes}}$ (ns) & $t_{\text{propagation}}$ (ns) & $T_{total}$ (ns) \\
\midrule
10 & 1 & 30 & 50 & 81 \\
1 & 1 & 30 & 5 & 36 \\
0.1 & 1 & 30 & 0.5 & 31.5 \\
0.01 & 1 & 30 & 0.05 & 31.05 \\
\bottomrule
\end{tabular}

\section{Channel Capacity and Effective Utilization}

\paragraph{Channel Capacity (Shannon):}
Shannon Capacity ($C$) is:
\[
C = B \log_2(1 + \text{SNR})
\]
For short copper cables (SNR typically $>30$dB):
\[C \approx 10 \times B\]

Thus, Ethernet line rate (10G/100G) sets practical throughput limits.

\paragraph{Effective Throughput:}
Considering RTT and Jumbo Frames (~9000 bytes $\approx$ 72,000 bits):

\begin{tabular}{@{}cccc@{}}
\toprule
Length (m) & RTT (ns) & RTT ($\mu$s) & Effective Throughput (Gbps) \\
\midrule
10 & 162 & 0.162 & 0.444 \\
1 & 72 & 0.072 & 1 \\
0.1 & 63 & 0.063 & 1.14 \\
0.01 & 62.1 & 0.0621 & 1.16 \\
\bottomrule
\end{tabular}

Single-packet throughput is RTT limited; real-world Ethernet uses pipelines.

\section{Pipeline (Multiple Packets In Flight)}

Packets needed to saturate a 10Gbps line with 9000-byte packets:
\[
\text{Packets in flight} = \frac{\text{Line Rate} \times \text{RTT}}{\text{Packet Size}}
\]

\begin{tabular}{@{}cc@{}}
\toprule
Length (m) & Packets in Flight \\
\midrule
10 & 23 \\
1 & 10 \\
0.1 & 9 \\
0.01 & 9 \\
\bottomrule
\end{tabular}

Shorter cables reduce pipeline depth and simplify buffer management.

\section{Summary of Results}
\begin{itemize}
	\item Propagation delays negligible below 1m.
	\item Utilization depends primarily on pipeline depth.
	\item Short cables (1cm, 10cm) achieve near-ideal throughput easily.
	\item Longer cables ($\geq$10m) require more buffering for full utilization.
\end{itemize}

\section{Conclusion}

For short-range Ethernet links (1m or less), intrinsic latencies dominate, making cable propagation negligible. With appropriate buffering and pipeline management, near-maximum Ethernet throughput is achievable, strongly favoring short-range interconnects for high-performance and low-latency Ethernet designs.


\end{document}
