\documentclass[HFT-main.tex]{subfiles} % NEW VERSION IS CASE.tex
\begin{document}
\section{Ethernet Overhead}

Standard Ethernet framing\marginnote{See IEEE 802.3 standard; also \textit{Ethernet: The Definitive Guide} by C. Spurgeon} includes:
\begin{itemize}
  \item \textbf{Preamble}: 7 bytes of \texttt{0x55} + 1 byte SFD
  \item \textbf{Destination MAC}: 6 bytes
  \item \textbf{Source MAC}: 6 bytes
  \item \textbf{Type/Length field}: 2 bytes
  \item \textbf{Payload}: 46--1500 bytes
  \item \textbf{Frame Check Sequence (FCS)}: 4 bytes
  \item \textbf{Inter-frame Gap (IFG)}: 12 bytes idle
\end{itemize}

In static point-to-point HFT environments, nearly all of this can be viewed as redundant overhead. When link endpoints and roles are fixed, headers and metadata convey no new information.

\section{Technical Feasibility}

Eliminating traditional Ethernet elements requires a different framing model---a \textit{contract} between hardware endpoints.

\subsection*{No Preamble / SFD}

The preamble synchronizes clocks in legacy systems. In modern SerDes links with embedded clock recovery\marginnote{Compare to PCIe, which uses 128b/130b encoding without explicit preambles}, this is already managed.

\textbf{Feasibility}: \emph{Yes}, with stable point-to-point SerDes links or slotted clocks.

\textbf{Tradeoff}: \emph{Reduced robustness}. In noisy environments, lack of framing risks misalignment or data corruption. This is acceptable in sealed environments like short DAC cables or on-board links.

\subsection*{No MAC Addresses}

HFT links are fixed. The source and destination are implicitly known.

\marginnote{Used in protocols like NVLink or Intel UPI, where links are directly mapped.}

\textbf{Feasibility}: \emph{Yes}, especially in sealed, controlled deployments.

\textbf{Tradeoff}: \emph{Loss of dynamic addressing or discovery}. Requires physical link provisioning. Cannot reroute or adapt without external orchestration.

\subsection*{No Inter-frame Gap}

IFGs were designed to give PHYs time to reset. In engineered environments using continuous signaling, these are unnecessary.

\marginnote{Similar techniques are used in FPGAs with streaming AXI interfaces or in 100G optical transport}

\textbf{Feasibility}: \emph{Yes}, if physical layer supports continuous or slotted framing.

\textbf{Tradeoff}: \emph{Thermal and electrical stress}. Removing idle periods may increase power draw or reduce link lifespan. Needs low-jitter precision timing.

\subsection*{No Headers at All}

Header fields can be eliminated if:
\begin{itemize}
  \item Frame size is known a priori or fixed
  \item Error detection is moved to another layer (e.g., optical layer or persistent memory logic)
  \item Transaction context is inferred
\end{itemize}

\textbf{Feasibility}: \emph{Yes}, in tightly coupled systems with fixed-length transactions.

\marginnote{Used in Time-Triggered Ethernet (TTE), and in DRAM command buses}

\textbf{Tradeoff}: \emph{Loss of introspection and self-description}. Makes debugging and monitoring difficult. May require separate control channel for diagnostics.

\section{Security and Fault Isolation}

This design assumes:
\begin{itemize}
  \item No dynamic routing or discovery
  \item No reconfiguration or rerouting
  \item Total operational control over both endpoints
\end{itemize}

\marginnote{Compare to InfiniBand RC mode: reliable connections with fixed endpoints and explicit path setup \href{https://docs.nvidia.com/networking/display/rdmaawareprogrammingv17/transport+modes}{RDMA Transport Modes}}

\textbf{Tradeoff}: \emph{Operational fragility}. Misconfiguration can lead to silent failure. But the simplification pays dividends in latency-critical deployments.

\section{Physical Layer Considerations}

Key enablers of this model include:
\begin{itemize}
  \item \textbf{SerDes-level framing} (e.g., 64b/66b)
  \item \textbf{Slotted time-based framing}
  \item \textbf{Sideband channels} for negotiation and integrity checking
\end{itemize}

\marginnote{See also: HFT deployments using cut-through switches and FPGA SmartNICs with bypass logic}

Each slot corresponds to a transaction. Bidirectional acknowledgments (e.g., Slot ACKs or ``SACKs'') can be used to indicate success or trigger reversal of actions.

\textbf{Reference}: Early reversible protocols in \textit{Hot Interconnects} (2005–2012) and dataflow routing in on-chip networks.

\section{Protocol Design Implications}

This minimalist interaction protocol:
\begin{itemize}
  \item Removes legacy structures optimized for routing and error correction
  \item Requires extreme pre-coordination of endpoints
  \item May use application-specific framing or alignment
  \item Demands an invariant frame format and timing agreement
\end{itemize}

\textbf{Tradeoff}: \emph{Loss of generality}. However, when the application is narrow (e.g., HFT price updates), the gains in determinism and efficiency vastly outweigh flexibility.

\section{Conclusion: Post-Ethernet}

This proposal represents a departure from Ethernet’s historical flexibility. It may at first seem to not be backwards compatible with 802.3, but compatibility is provided by the escape bits and codes), and the the Address fields can be compatible, not only with the two 48-bit MAC Addresses, but the 32-Bit IP and 64-Bit IP address also. This is just a matter of mapping the encoding from the \texttt{encoding} field.

For HFT, or similar environments, it represents a return to the core of information transmission: symbols sent from A to B with zero ambiguity.

\begin{quote}
\textit{Ethernet began as a way to share a cable. Today, it must become a way to share certainty.}\marginnote{\textit{Would Shannon approve?} Perhaps. But only if we prove we need fewer bits to transmit the same certainty.}
\end{quote}

Removing headers, gaps, addresses, and preambles is not only possible—it is essential for ultra-fast applications where every bit counts.\marginnote{\textit{Would Metcalfe approve?} Perhaps. But only if we coordinate with the current Ethernet consortia including 802?}

This proposal captures the main idea of using the minimalist approach to transporting data over copper or fiber links. % starting with SFP+ / SFP-25 Ethernet PHY.

HFT payloads can be as small as a bit to confirm that a price level was hit, more practically a few bytes to encode a message, or tens of bytes to convey market data or orders.

%One good example of very small HFT data is from a company called \href{https://raft-tech.com}{RAFT Technologies} that uses shortwave HAM radio bands to send market data around the circumference of the globe with minimal latency.   The messages they send are a few bytes to tens of kB.

%\input{History-Content}

\end{document}
