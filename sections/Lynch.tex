\documentclass[../HFT-main.tex]{subfiles} % NEW VERSION IS CASE.tex
\begin{document}

%\maketitle
\section{Lynch Model}
Quoted  from Bill Lynch's Email (02025-MAR-XX):
%1)  As per the comments in my last message - 
\begin{quotation}
\begin{enumerate}
\item ``It is quite possible to use a higher level protocol ON TOP of an ACK/Nak protocol to achieve throughput approaching the connection's inherent throughput capacity.  Basically such a protocol implementation should:

\begin{enumerate}
	\item De-multiplex on a round-robin basis, the input stream into several sub-streams.
	\item Transmit all sub-streams independently and concurrently using separate instances of an ACK/NAK protocol.  The actual ACK/NAK protocol may even vary from sub-stream instance to instance. 
	\item At the receiving end re-multiplex the sub-streams back into a received copy of the original .
\end{enumerate}

Since the bandwidth capacity of the connection will be finite, there will be limit to the number of sub-streams that will yield an significant improvement.  This approach is not new - it is fundamentally the same as the flow control in TCP.  It is only a question of the size and representation of the system state.

%\begin{framed}
I think it is quite untrue that ACK/NAK is a severe and fundamental limit on transmission efficiency.
%\end{framed}

\item  I do believe that Ethernet's primary asset is the large and thriving ecosystem that allows a multitude of vendors to contribute to a customer system.  That ecosystem was grown by a free, open, and detailed system of standards provided first by the involved vendors and later by the professional standards association.  I can't tell you how many improvements proposals emerged from even within those vendors - proposals that improved the system performance by a few percent at the cost of compatibility. 

A bit more subtle was the substantial effort (beginning with the BlueBook and through the professional standards)  provided not only the nominal value of parameters but also carefully considered tolerances constructed and analyzed to assure system operation when the system had been constructed of modules provided by various vendors (an original 2 user Ethernet could have equipment from 7 different vendors).  This original DIX demonstration at the 1981 NCC in Chicago's McCormick Place was aimed directly at this.

There were a number of commercial Ethernet networking vendors with offered systems that worked all the time when composed of components from that vendor (and some but not all of the time when composed of components from mixed vendors).  The vendors either found their addressed market constrained or were force to remove the network system from the product line.''

\end{enumerate}
\end{quotation}

\end{document}