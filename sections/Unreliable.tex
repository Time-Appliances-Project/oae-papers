\documentclass[../HFT-main.tex]{subfiles}
\title{Why Ethernet Needs to Be Reliable}


\begin{document}
% ---------- Metadata ----------
\section{Why Ethernet Needs to Be Reliable}


% ---------- Helper macro ----------
\marginnote{An unreliable link can never solve congestion; %
it can at best conceal it, and at worst amplify it.}
 

%%%%%%%%%%%%%%%%%%%%%%%%%%%%%%%%%%%%%%%%%%%%%%%%%%%%%%%% QUESTION
\section*{Question}
Why Ethernet needs to be reliable.  Distinguish between \emph{imposition}
networks and \emph{promise} networks (Promise Theory).  Distinguish between
``useful packets that are acknowledged'' and those that ``overload (a) the
receiver (which drops them) and the network that drops them)''.  Go deeply
into the problem of congestion, and how unreliable networks are a major
cause of our inability to solve the congestion problem, because they make
the problem worse, not better.  Bring in unique perspectives on how
\emph{unpaired} information can only be correlations, not reliable
delivery.

%%%%%%%%%%%%%%%%%%%%%%%%%%%%%%%%%%%%%%%%%%%%%%%%%%%%%%%% ESSAY
\section{Introduction}
Ethernet's spectacular success for fifty years rests on three pillars:
ubiquity, low cost, and \emph{best--effort} simplicity.  But the assumption
that ``bit errors are rare and losses are somebody else's problem'' has
aged badly.  Modern workloads---RDMA storage, NVMe--oF, micro--transactions
in high--frequency trading, chiplet fabrics with sub--microsecond
deadlines---have turned ``good enough'' into ``not nearly enough''.

\section{Imposition vs.\ Promise Networks}

\subsection{Imposition networks}
An \emph{imposition} network forces state upon its peers.  Classical
Ethernet is such a system: a transmitter imposes frames on the medium and
presumes they will be carried.  If a switch, buffer, or receiver cannot
cope, frames are silently discarded.  The sender remains ignorant until a
higher layer notices the loss.

\subsection{Promise networks}
A \emph{promise} network inverts the relationship: each agent voluntarily
advertises what it will accept and under what constraints.  Congestion
feedback, explicit flow control, and lossless credit schemes make the
fabric a living contract.  InfiniBand, Fibre Channel, and CAN are promise
networks by design; PFC--enabled RoCE grafts promise semantics onto
Ethernet with mixed success.

\section{Useful vs.\ Harmful Packets}

\begin{enumerate}
  \item \textbf{Paired / Acknowledged frames}: every frame is matched by a
  handshake or credit and carries information \emph{with commitment}.
  \item \textbf{Unpaired / Blind frames}: sent speculatively or in excess of
  credit, they eat capacity, fill buffers, and are eventually dropped,
  contributing only to congestion.
\end{enumerate}

\section{Congestion: Why Unreliability Amplifies It}

\subsection{Positive feedback loops}
Classic Ethernet plus TCP uses \emph{loss} as its congestion signal, but
the signal arrives one--three RTTs late.  A 100~Gb/s host can inject
millions of frames during that blind window:
\[
  \text{loss} \;\longrightarrow\; \text{retransmit} \;\longrightarrow\;
  \text{more traffic} \;\longrightarrow\; \text{more loss}.
\]

\subsection{Bufferbloat}
Deeper switch buffers merely store congestion instead of preventing it; the
queueing delay often dwarfs propagation delay, wrecking tail--latency
guarantees.

\subsection{Head--of--line blocking}
Dropped frames arrive out of order; completion time is dominated by
recovery, not transmission.  A link--layer ACK/NAK guarantee keeps delivery
in order and prevents the stall.

\section{Unpaired Information Is Only Correlation}
A frame that never arrives does not increase mutual information; it leaves
a correlation hole.  Reliable Ethernet lets the link itself guarantee that
every delivered frame is the one that was sent, exactly once.

\section{Design Implications}
\begin{itemize}
  \item \textbf{Credit--based losslessness}: lightweight link--local
  ACK/NAK or credit return bounds outstanding data without InfiniBand
  complexity.

  \item \textbf{Actionable ECN}: with no drops, an ECN mark is a clear,
  immediate promise that congestion exists here and now.

  \item \textbf{Deterministic latency}: bounded jitter enables
  hardware--timestamped transactions and reversible protocols.

  \item \textbf{Energy efficiency}: eliminating retransmissions and deep
  buffering cuts joules per useful bit.
\end{itemize}

\section{Conclusion}
Cheap and cheerful won in 1973, but the economics have flipped: every lost
packet spawns expensive silicon, energy, and complexity at higher layers.
Turning Ethernet from an \emph{imposition} fabric into a \emph{promise}
fabric breaks the congestion feedback loop, converts blind correlations
into committed information, and opens the next chapter for the world's most
enduring network.  Reliable Ethernet is no luxury; it is the prerequisite
for solving the congestion pathologies that threaten Ethernet's future.

\end{document}
